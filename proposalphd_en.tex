%%%%%%%%%%%%%%%%%%%%%%%%%%%%%%%%%%%%%%%%%
% Simple Article
% Integrated article template with simple for make4ht
% LaTeX Class
% Version 1.0 (10/11/20)
%
% This class originates by:
% Vel and  Nicolas Diaz
%
% Authors:
% Muhammad Uliah Shafar
%
%
% Free License:
%
%
%%%%%%%%%%%%%%%%%%%%%%%%%%%%%%%%%%%%%%%%%
\documentclass[11pt]{simart} % Font size (can be 10pt, 11pt or 12pt)

%----------------------------------------------------------------------------------------
%	TITLE SECTION
%----------------------------------------------------------------------------------------
% MAIN TITLE SECTION
\title{
\textbf{Waterfront Public Space Characteristic In Shaping Indentity of Parepare City} \\
\textbf{{Research Proposal \\}}
} % Title and subtitle
%\date{\textbf{\DTMtoday}}
\date{\textbf{\today}}
\author{Uliah Shafar}

%----------------------------------------------------------------------------------------
% OTHER TITLE SECTION

%\title{\textbf{Sistem Sarana dan Prasarana Jl. Pinggir Laut} \\ {\Large\itshape Infrastructure of Waterfront Parepare City}} % Title and subtitle

%\author{\textbf{Uliah Shafar} \\ \textit{Universitas Diponegoro}} % Author and institution

%\date{\today} % Date, use \date{} for no date

%----------------------------------------------------------------------------------------



\begin{document}

\maketitle % Print the title section

%----------------------------------------------------------------------------------------
%	ABSTRACT AND KEYWORDS
%----------------------------------------------------------------------------------------
\begin{comment}

\begin{abstract}
Lorem ipsum dolor sit amet, consectetur adipiscing elit. Curabitur eget faucibus dolor. In posuere, est nec mollis ultrices, ante arcu tristique odio, et rhoncus tortor enim vitae lectus. Aenean auctor enim tempor risus vulputate finibus. Ut quis molestie ex, ut fringilla mauris. Suspendisse ornare sapien nec neque placerat dignissim. Sed vehicula feugiat dolor et blandit. Maecenas convallis diam a lacus faucibus faucibus. Quisque efficitur velit quis lorem consectetur, ac dictum est egestas.

\end{abstract}

\end{comment}
\hspace*{3.6mm}\textit{Keywords:} Lorem, Ipsum % Keywords

\vspace{30pt} % Vertical whitespace between the abstract and first section

%----------------------------------------------------------------------------------------
%	ESSAY BODY
%----------------------------------------------------------------------------------------
\section{Pendahuluan}

Public space has become hot discussion in city planning for the past 100 years. The increasing population cause there is a demand for public space. This increasing was depicted by the existence of the  residence's mass construction, dense vehicle in the road, and the increasing of business and economy sector. This increasing population stress every each individual in experience their daily life.
People then are looking for a restorative place together to escape from hustle and bustle of daily routines and get rid of the stress caused by ambience of the city.

Public space is a place that accommodate all kind of personal or group activities in searching of amusement and other positive things \citep{sadat2012}. \cite{sadat2012} argue that park, square, waterfront and street are a form from public space. Based on its function, public space can become a place for recreation, work, business, social culture and heritage.

People begin to considering public space when they look for a place that is accessible. This space become accommodation for people's daily activities. For example street which people cross everyday. Square that is used for routine exercise. The existence acitivities in public space contribute to city's life quality and reflect to adjacent people's condition.

\cite{ahmadi2009} state that public space own various characters. Those characters explain about what is in the physical and social environment \citep{dougherty2006}. In addition, \cite{hartanti2014} state that physical element, acitivity and ambience are the character's maker. Furthermore, few of researcher have found characteristic is a request of fulfillment and reflection of the circumstance of nearby people.

Parepare have plenty public space with each own characteristic. For example, Andi Makassau Park, Parepare beach, Syariah Park, Tonrangeng waterfront, and etc. Now, all of the attention is in the waterfront public space which is the focus of the research.

Solo karajae Festival represent the heart of Parepare waterfront is still beating. A thousand people come together to tonrangeng to watch a variety of shows in a week. This public space, according to \cite{hartanti2014}, explain that the identity of the city through its physical element. It is because people enjoy the city by sightseing, feeling, and understanding the information such as paths, landmarks, trees, sea, and other elements.

Sometimes the relationship between public space and city are very closed. People mark that they have been in the Parepare city by taking photos in the BJ Habibie public space. Public space, its name, its characteristic, have become the city identity. Therefore, space elements had better be well designed in order to show the right characters and perceive by users \citep{hartanti2014}.

A number of public space stimulate Parepare to carry ideas of The City of BJ. Habibie's Love. BJ. Habibie was a third president of Indonesia who was born in Parepare. His Contribution toward Indonesia has become inspiration to Parepare city. This ideas should be the new identity of Parepare.
A identity of place, based on Kevin Lynch, is a foundation to recognize something to be a discrete things \citep{hartanti2014}. According to \textit{Webster's Ninth New Collegiate}, Identity is the distinctive character or personality of a things or individual. Similarly, according to \cite{hartanti2014}, the city identity can be formed through characteristic


the same, according to \cite{hartanti2014}, city's identity can be formed through characters which its inside. It can through physical form or physical appearance. However, it can go through the image ability concept which consist of identity, structure, and meaning \citep{lycnh1984}.

\cite{hartanti2014} state that identity can become a requirement of good environment. She added that form identity of city will strengthening its difference, building an event or specific mark and supporting few specific motive. For example Salo Karajae festive which become unique and special event on the seaside. That event has been ongoing for the last years and succeed. Many different attraction was conducted such as video mapping on the sea, beautify boat, and others activity.

The research about public space character has been broadly conducted, however little has been known conducted in the seaside area. The appearance of difference characteristic because of its border close to the sea is an extra value toward public space.
\cite{hussein2014} explain that the succesfull of the city attract people to the seaside depend on public space there. Therefore, this research will explore the characteristic of public space in shaping the identity of Parepare city.

\section{Rumusan Masalah}

A successful city identity came from a strong and free image (Oktay,2002). Most of it is characterized by public space. Meanwhile, the most prominent public space is currently in the coastal area. This region is an amazing and flexible place (Hussein,2014). According to Oktay(2002), in the seeking for city identity in the changing city context, finding public spaces should include broad notions such as people's activities, experiences, and relationships with the public space itself.
Coastal areas of the sea do have extraordinary potential. Parepare as a city which bordering of the sea takes advantage of this opportunity to build a number of seacoast public spaces. This development is considered drastic, where within the span of a period of political office, more than one public space is produced and reformed. The development first took place in 2011. It renovated the Senggol Beach public space which is now called Parepare Beach. This is followed by the construction of other public spaces from scratch. In addition, development also spread to the city center.
Unlike other cities, Parepare departs with an undeveloped identity. Many cities have had a strong identity for a long time. As an example, the city of Bogor, a domestic city, had a city identity related to plants (Hartanti and Martokusumo,2014). In addition, there is a city of north Cyprus, a foreign city, had the identity of the old city (Oktay,2002).Oktay(2002) explained that identity consists of a number of characteristics or elements of public space that can be identified.
The identity of the city of Parepare which is promoted recently is the city of love. That triggered the construction of a public space called BJ Habibie and Ainun. One of the most prominent is the BJ Habibie monument which islocated in the Andi Makkasau park. Two of them are located on the sea coast, namely the multipurpose building and the BJ Habibie hospital which is close to the Tonrangeng riverside. Even though this development implied BJ Habibie's “city of love” with a number of naming or public statues in public spaces, it did not mean that public spaces can automatically form the expected urban identity.
The characteristics of each public space played an important role in shaping urban identity (Oktay,2002).Lynch(1984) described the character of public space by paying attention to the elements that are arranged and their relationship to each other and forms a distinguishing character, which indicated an identity (Hartanti and Martokusumo,2014). Hartanti and Martokusumo(2014) emphasized that differentiating is the most important process in recognizing a place identity. This related to physical form, activity and atmosphere. Based on the previous description, this research studied the character of public space on the sea coast. So this research would answer a number of research questions as follows:
\begin{enumerate}
\item What is the current identity of the city of Parepare and its influence for visitors?
\item What is the distinguishing character between the seacoast public spaces that form the identity of the city of Parepare?
\end{enumerate}

\section{Objective of the research}

This study aims to study the characteristics of seacoast public spaces. These characters are explained in the context of spatial elements, people's activities, experiences, and relationships with public spaces. So that this research can explain the identity of the city of Parepare. According to Oktay(2002), identity is able to build individual interest in a place and stay for a long time. This ability is a sign of a successful coastal public space (Hussein,2014).
The researcher hoped that the results of this study can be used to identify the identity of the city of Parepare. That way, the development of parepare cities can refer to identity so that it gets stronger.
\begin{enumerate}
\item To understand the city's current identity and its impact on visitors.
\item To study the differentiating characters between seacoast public spaces in forming city identities.
\end{enumerate}

\section{Research method}
This study aims to find out the differentiating characters between spaces in forming city identities. In terms of achieving these goals, a research approach is needed. The research approach is a process or research plan including steps, detailed methods in the collection, analysis, meaning of the findings (Creswell and Poth,2016). This research approach is qualitative and quantitative (mixed-method).
A qualitative approach produces an overall picture of a phenomenon through direct understanding and observation of the phenomenon in the research object accompanied by data sources (Creswell and Poth,2016). This phenomenon is in the form of characteristics of public spaces that form the identity of public spaces. According toGroats and Wang(2013), qualitative research emphasizes the inductive process, asCreswell and PothExplain this by giving open-ended questions. The question is in the form of what identity the individual feels in the parepare city and how does that affect the individual.Creswell and Poth(2016) explained that this question can develop along with the researcher's understanding of the problems in the field (Creswell and Poth,2016). The quantitative approach is used to examine certain populations or samples. Data collection based on research variables. Then administering a questionnaire survey to find the most dominant characteristics for users in seacoast public spaces. So that this research can find out the identity of seacoast public space through its forming characteristics for better design.

\section{Result of the research}
The rapid development of a city can lead to the failure of the formation of a city's identity. More than that, you can also remove and change existing characters (Hartanti and Martokusumo,2012). People can recognize a city as something different if they get the essence of the city's identity. According toOktay(2002), a strong identity in a city will attract and support the people in a city. Public space, accordingHartanti and Martokusumo(2014), explains the identity of a city through its physical characteristics. Through this research, the characteristics of a number of public spaces on the coast that form the city's identity can be identified. So that urban planning can be based on these characteristics and follow known identities.

\begin{comment}
\section{Subject of study}
In the past few years, I have been studying about public space topic especially in the waterfront. It has resulted one of thesis entitled Visitor Preference on Senggol Waterfront of Parepare and a publication related to public realms. The reason I studied it was that there are an increasing of demand and inftrastructure development across the public space waterfront in my city.
\end{comment}

%\begin{figure}[htbp]
%\centering
%\begin{subfigure}{6cm}
%\centering\includegraphics[width=5cm]{placeholder.jpg}
%\caption{}
%\end{subfigure}%
%\begin{subfigure}{6cm}
%\centering\includegraphics[width=5cm]{placeholder.jpg}
%\caption{}
%\end{subfigure}\vspace{10pt}
%
%\caption{Lorem Ipsum}
%\label{}
%\end{figure}

%----------------------------------------------------------------------------------------
%	BIBLIOGRAPHY
%----------------------------------------------------------------------------------------
\bibliographystyle{apalike}
\bibliography{biblio.bib}
\end{document}
