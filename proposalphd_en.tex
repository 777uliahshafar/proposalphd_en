%%%%%%%%%%%%%%%%%%%%%%%%%%%%%%%%%%%%%%%%%
% Simple Article
% Integrated article template with simple for make4ht
% LaTeX Class
% Version 1.0 (10/11/20)
%
% This class originates by:
% Vel and  Nicolas Diaz
%
% Authors:
% Muhammad Uliah Shafar
%
%
% Free License:
%
%
%%%%%%%%%%%%%%%%%%%%%%%%%%%%%%%%%%%%%%%%%
\documentclass[11pt]{simart} % Font size (can be 10pt, 11pt or 12pt)

%----------------------------------------------------------------------------------------
%	TITLE SECTION
%----------------------------------------------------------------------------------------
% MAIN TITLE SECTION
\title{
\textbf{Waterfront Public Space Characteristic In Shaping Indentity of Parepare City} \\
\textbf{{Research Proposal \\}}
} % Title and subtitle
%\date{\textbf{\DTMtoday}}
\date{\textbf{\today}}
\author{Uliah Shafar}

%----------------------------------------------------------------------------------------
% OTHER TITLE SECTION

%\title{\textbf{Sistem Sarana dan Prasarana Jl. Pinggir Laut} \\ {\Large\itshape Infrastructure of Waterfront Parepare City}} % Title and subtitle

%\author{\textbf{Uliah Shafar} \\ \textit{Universitas Diponegoro}} % Author and institution

%\date{\today} % Date, use \date{} for no date

%----------------------------------------------------------------------------------------



\begin{document}

\maketitle % Print the title section

%----------------------------------------------------------------------------------------
%	ABSTRACT AND KEYWORDS
%----------------------------------------------------------------------------------------
\begin{abstract}
Lorem ipsum dolor sit amet, consectetur adipiscing elit. Curabitur eget faucibus dolor. In posuere, est nec mollis ultrices, ante arcu tristique odio, et rhoncus tortor enim vitae lectus. Aenean auctor enim tempor risus vulputate finibus. Ut quis molestie ex, ut fringilla mauris. Suspendisse ornare sapien nec neque placerat dignissim. Sed vehicula feugiat dolor et blandit. Maecenas convallis diam a lacus faucibus faucibus. Quisque efficitur velit quis lorem consectetur, ac dictum est egestas.

\end{abstract}

\hspace*{3.6mm}\textit{Keywords:} Lorem, Ipsum % Keywords

\vspace{30pt} % Vertical whitespace between the abstract and first section

%----------------------------------------------------------------------------------------
%	ESSAY BODY
%----------------------------------------------------------------------------------------
\section{Pendahuluan}

Public space has become hot discussion in city planning for the past 100 years. The increasing population cause there is a demand for public space. This increasing was depicted by the existence of the  residence's mass construction, dense vehicle in the road, and the increasing of business and economy sector. This increasing population stress every each individual in experience their daily life.
People then are looking for a restorative place together to escape from hustle and bustle of daily routines and get rid of the stress caused by ambience of the city.

Public space is a place that accommodate all kind of personal or group activities in searching of amusement and other positive things \citep{sadat2012}. \cite{sadat2012} argue that park, square, waterfront and street are a form from public space. Based on its function, public space can become a place for recreation, work, business, social culture and heritage.

People begin to considering public space when they look for a place that is accessible. This space become accommodation for people's daily activities. For example street which people cross everyday. Square that is used for routine exercise. The existence acitivities in public space contribute to city's life quality and reflect to adjacent people's condition.

\cite{ahmadi2009} state that public space own various characters. Those characters explain about what is in the physical and social environment \citep{dougherty2006}. In addition, \cite{hartanti2014} state that physical element, acitivity and ambience are the character's maker. Furthermore, few of researcher have found characteristic is a request of fulfillment and reflection of the circumstance of nearby people.

Parepare have plenty public space with each own characteristic. For example, Andi Makassau Park, Parepare beach, Syariah Park, Tonrangeng waterfront, and etc. Now, all of the attention is in the waterfront public space which is the focus of the research.

Solo karajae Festival represent the heart of Parepare waterfront is still beating. A thousand people come together to tonrangeng to watch a variety of shows in a week. This public space, according to \cite{hartanti2014}, explain that the identity of the city through its physical element. It is because people enjoy the city by sightseing, feeling, and understanding the information such as paths, landmarks, trees, sea, and other elements.

Sometimes the relationship between public space and city are very closed. People mark that they have been in the Parepare city by taking photos in the BJ Habibie public space. Public space, its name, its characteristic, have become the city identity. Therefore, space elements had better be well designed in order to show the right characters and perceive by users \citep{hartanti2014}.

A number of public space stimulate Parepare to carry ideas of The City of BJ. Habibie's Love. BJ. Habibie was a third president of Indonesia who was born in Parepare. His Contribution toward Indonesia has become inspiration to Parepare city. This ideas should be the new identity of Parepare.
A identity of place, based on Kevin Lynch, is a foundation to recognize something to be a discrete things \citep{hartanti2014}. According to \textit{Webster's Ninth New Collegiate}, Identity is the distinctive character or personality of a things or individual. Similarly, according to \cite{hartanti2014}, the city identity can be formed through characteristic


the same, according to \cite{hartanti2014}, city's identity can be formed through characters which its inside. It can through physical form or physical appearance. However, it can go through the image ability concept which consist of identity, structure, and meaning \citep{lycnh1984}.

\cite{hartanti2014} state that identity can become a requirement of good environment. She added that form identity of city will strengthening its difference, building an event or specific mark and supporting few specific motive. For example Salo Karajae festive which become unique and special event on the seaside. That event has been ongoing for the last years and succeed. Many different attraction was conducted such as video mapping on the sea, beautify boat, and others activity.

The research about public space character has been broadly conducted, however little has been known conducted in the seaside area. The appearance of difference characteristic because of its border close to the sea is an extra value toward public space.
\cite{hussein2014} explain that the succesfull of the city attract people to the seaside depend on public space there. Therefore, this research will explore the characteristic of public space in shaping the identity of Parepare city.

\section{Rumusan Masalah}

The successfull of the city identity was from the freedom and strenght of imagebility \citep{oktay2002}. Mostly it was characterised by public space. While the most dominant of public space nowadays is in the seaside.
This area is a flexible and astonishing place \citep{hussein2014}. According to \cite{oktay2002}, in searching of city identity in context of changing city, found that a space should be consist of broad meaning like acitivity, experience, and relationship with public space itself.

The area of seaside indeed have extraordinary potency. Parepare as city that border with the sea use this chance to construct number of waterfront public space. This construction is said drastic where it











%\begin{figure}[htbp]
%\centering
%\begin{subfigure}{6cm}
%\centering\includegraphics[width=5cm]{placeholder.jpg}
%\caption{}
%\end{subfigure}%
%\begin{subfigure}{6cm}
%\centering\includegraphics[width=5cm]{placeholder.jpg}
%\caption{}
%\end{subfigure}\vspace{10pt}
%
%\caption{Lorem Ipsum}
%\label{}
%\end{figure}

%----------------------------------------------------------------------------------------
%	BIBLIOGRAPHY
%----------------------------------------------------------------------------------------

\bibliographystyle{apalike}

\bibliography{biblio.bib}



\end{document}
